\documentclass{beamer}
\usepackage{relsize}
\usepackage{color}

\usepackage{listings}
\usetheme{CambridgeUS}
%\usepackage{beamerthemesplit} % new 
\usepackage{enumitem}
\usepackage{amsmath}                    % See geometry.pdf to learn the layout options. 
\usepackage{amsthm}                   % See geometry.pdf to learn the layout options. There 
\usepackage{amssymb}                    % See geometry.pdf to learn the layout options. 
\usepackage[utf8]{inputenc} 
\usepackage{graphicx}
\usepackage[english,bulgarian]{babel}
\usepackage[framemethod=tikz]{mdframed}
\usepackage{caption}
\usepackage{tikz}
\usepackage{forest}
\usetikzlibrary{shapes,arrows,positioning,calc,chains}

\lstset{language=C++,
                basicstyle=\ttfamily,
                keywordstyle=\color{blue}\ttfamily,
                stringstyle=\color{red}\ttfamily,
                commentstyle=\color{green}\ttfamily,
                morecomment=[l][\color{magenta}]{\#}
}

\newtheorem{mydef}{Дефиниция}[section]
\newtheorem{lem}{Лема}[section]
\newtheorem{thm}{Твърдение}[section]

\DeclareMathOperator{\restrict}{\upharpoonright}

\setitemize{label=\usebeamerfont*{itemize item}%
  \usebeamercolor[fg]{itemize item}
  \usebeamertemplate{itemize item}}

\setbeamercovered{transparent}

\captionsetup{font=footnotesize}

\lstset{breaklines=true}
\tikzset{
block/.style = {draw, fill=white, rectangle,align = center},
entry/.style = {draw, fill=black, circle, radius=3em},
condition/.style = {draw, fill=white, diamond, align = center,node distance=3cm},
fork/.style = {draw, fill=black, circle,inner sep=1pt},
lnode/.style={rectangle split, rectangle split parts=3,draw, rectangle split horizontal},
treenode/.style = {align=center, inner sep=0pt, text centered, circle, font=\sffamily\bfseries, draw=black, fill=white, text width=1.5em}
}


\begin{document}
\title[Обектно-ориентирано програмиране]{Обработка на изключения} 
\author{Калин Георгиев} 
\frame{\titlepage} 


\section{Изключения} 


\begin{frame}[fragile]
  \frametitle{Механизъм на изключенията}

  \begin{itemize}
    \item Алтернатива на ``нормалния резултат''
    \item Само за ситуации, които биха нарушили ``нормалната'' програмна логика
    \item Позволяват ``възстановяване от грешки''
  \end{itemize}

  \bigskip

  Какво е ``Грешка''? Защо стават грешки?
    
  \end{frame}

  
  \begin{frame}[fragile]
    \frametitle{Елементи на изключенията}
  
    \begin{itemize}
      \item Възникване: \texttt{throw}
      \item Следене: \texttt{try}
      \item Прихващане: \texttt{catch}
    \end{itemize}
  \end{frame}
  

\begin{frame}[fragile]
  \frametitle{throw}

  \begin{itemize}
    \item Повече от един \texttt{throw} във функция
    \item Тип на изключението
  \end{itemize}
\end{frame}

\begin{frame}[fragile]
  \frametitle{catch}

  \begin{itemize}
    \item Тип на изключението
    \item Различни \texttt{catch} клаузи за различните типове изключения
    \item std::exception
  \end{itemize}
\end{frame}


\begin{frame}[fragile]
  \frametitle{Жизнен цикъл на обектите при изключения}

  \begin{itemize}
    \item \texttt{Stack unwinding}
    \item Изключения в конструктори
    \begin{itemize}
        \item RAII: Притежанието на ресурс е инварианта на класа
    \end{itemize}    
    \item Изключения в деструктори
    \begin{itemize}
      \item Изключение по време на обработка на изключение
      \item Прехвърляне на отговорността на потребителя (напр. std::fstream::close)
    \end{itemize}
  \end{itemize}

\end{frame}


\begin{frame}[fragile]
  \frametitle{Алтернативи във функционален стил}

  \begin{itemize}
    \item \verb#std::pair<T,bool>#
    \item \verb#std::optional<T>#
  \end{itemize}
\end{frame}

\begin{frame}
\centerline{Благодаря за вниманието!}
\end{frame}


\end{document}

\begin{columns}[t]
  \begin{column}{0.55\textwidth}

  \end{column}
  \begin{column}{0.45\textwidth}

  \end{column}
\end{columns}


